\documentclass{protokol}
%------------------- Zde vyplňte údaje -------------------------------
\autor{Václav Horáček}
\autorID{256296}
\autorr{Jan Holík}
\autorrID{256295}
\rocnik{3}
\merenodne{23.\,9.\,2025}
\nazev{Měření otáček}
\predmet{Snímače}
%=====================================================================
\begin{document}
\maketitle                  % Vygeneruje titulní stránku podle vyplněných údajů
%\tableofcontents\newpage   % Vygeneruje obsah
%------------------- Zde začíná samotný dokument ---------------------

\section{ZADÁNÍ}\label{kap:zadani}
    \begin{enumerate}
        \item   Změřte a vyneste do grafu závislost výstupního napětí tachodynama na otáčkách
                v rozsahu ±2000 ot/min. Určete pomocí MNČ konstantu K tachodynama a porov-
                nejte ji s údaji výrobce (vypočítejte relativní odchylku). Určete linearitu. Nejistotu
                konstanty K určete ze dvou měřených bodů (pro tyto dva body hodnotu otáček
                změřte pomocí čítače).

        \item   Určete počet lamel komutátoru tachodynama.
                
        \item   U fotoelektrického odrazového snímače stanovte kolik impulzů připadá na jednu
                otáčku. Na čem závisí tato hodnota? Je možné na daném přípravku dosáhnout
                různých výsledků? Podmínky měření si zaznamenejte!
        
        \item   Na osciloskopu si prohlédněte a zaznamenejte tvar výstupních impulzů indukčního
                snímače a Hallovy sondy pro levé ozubené kolo. Průběh si zakreslete spolu s prů
                během vzdálenosti čela snímače od ozubeného kola tak, aby byla patrná souvislost
                výstupního signálu s tvarem ozubeného kola. Kdy se indukuje napětí na výstupu
                snímačů?

        \item   Zaznamenejte průběh signálů pro různé typy ozubených kol, včetně integrace. Jak
                souvisí tvar zubu a průběh integrálu výstupního napětí? U kterého tvaru zubu lze
                rozlišit směr otáčení?

        \item   Zobrazte na osciloskopu výstupní signál z optického inkrementální snímače a kva-
                draturního dekodéru pro oba směry otáčení. Průběhy si zaznamenejte (důležitá je
                fáze signálů) a zhodnoťte, jak se projeví změna směru na výstupních signálech. U
                kvadraturního dekodéru určete, v jakém módu pracuje (x1, x2 nebo x4). Srovnejte
                s teoretickými předpoklady.
                
        \item   Určete rozlišení inkrementálního optického snímače (počet impulzů na jednu otáčku)
                pomocí čítače.
                
        \item   Změřte efektivní hodnotu výstupních napětí resolveru v závislosti na úhlu nato-
                čení v rozsahu 0 až 360$^\circ$. Pro oba výstupy stanovte body, ve kterých se mění fáze
                vzhledem k budícímu signálu Uref . V intervalech vymezených těmito body změřte,
                má-li signál souhlasnou nebo opačnou fázi. Naměřená napětí vyneste do grafu. Fázi
                v grafu rozlište znaménkem (opačná = záporné). Z naměřených napětí vypočtěte
                úhel natočení a vyčíslete chybu v procentech z rozsahu. Změřte pracovní frekvenci
                resolveru (Uref ).

        \item   Na přípravku nastavte otáčky 900 ot/min, stroboskopem určete přesnou hodnotu a
                vypočítejte relativní odchylku.
    \end{enumerate}

\section{Úkol 1 - Převodní charakteristika tachodynama}
    \subsection{Teoretický rozbor}
    \subsection{Použité přístroje a přípravky}
    \subsection{Postup měření}
    \subsection{Naměřené hodnoty}
    \subsection{Zpracované výsledky měření}

\pagebreak

\section{Úkol 2 - Lamely tachodynama}
    \subsection{Teoretický rozbor}
    \subsection{Použité přístroje a přípravky}
    \subsection{Postup měření}
    \subsection{Naměřené hodnoty}
    \subsection{Zpracované výsledky měření}

\pagebreak

\section{Úkol 3 - Odrazový snímač}
    hello
    \subsection{Teoretický rozbor}
    \subsection{Použité přístroje a přípravky}
    \subsection{Postup měření}
    \subsection{Naměřené hodnoty}
    \subsection{Zpracované výsledky měření}

\pagebreak

\section{Úkol 4 - Indukční snímač/Hallova sonda}
    \subsection{Teoretický rozbor}
    \subsection{Použité přístroje a přípravky}
    \subsection{Postup měření}
    \subsection{Naměřené hodnoty}
    \subsection{Zpracované výsledky měření}

\pagebreak

\section{Úkol 5 - Průběhy signálů z měření ozubených kol}
    \subsection{Teoretický rozbor}
    \subsection{Použité přístroje a přípravky}
    \subsection{Postup měření}
    \subsection{Naměřené hodnoty}
    \subsection{Zpracované výsledky měření}

\pagebreak


\section{Úkol 6 - Inkrementálního optický snímač/Kvadratutní dekodér}
    \subsection{Teoretický rozbor}
    \subsection{Použité přístroje a přípravky}
    \subsection{Postup měření}
    \subsection{Naměřené hodnoty}
    \subsection{Zpracované výsledky měření}

\pagebreak


\section{Úkol 7 - Rozlišení inkrementálního optického snímače}
    \subsection{Teoretický rozbor}
    \subsection{Použité přístroje a přípravky}
    \subsection{Postup měření}
    \subsection{Naměřené hodnoty}
    \subsection{Zpracované výsledky měření}

\pagebreak
\section{Úkol 8 - Výstupní napětí resolveru}
    \subsection{Teoretický rozbor}
    \subsection{Použité přístroje a přípravky}
    \subsection{Postup měření}
    \subsection{Naměřené hodnoty}
    \subsection{Zpracované výsledky měření}

\pagebreak


\section{Úkol 9 - Měření otáček stroboskopem}
    \subsection{Teoretický rozbor}
    \subsection{Použité přístroje a přípravky}
    \subsection{Postup měření}
    \subsection{Naměřené hodnoty}
    \subsection{Zpracované výsledky měření}

\pagebreak


\section{Závěr}
\pagebreak
\section{Použitá literatura}


%------------------------------ Ukázky -------------------------------
\begin{comment} % Komentář bloku
    \newpage % Začátek na nové stránce
        
    % Ukázka obrázku
    \begin{figure}[!h]
        \centering
        \includegraphics[scale=0.3]{UAMT_color_CMYK_CZ.pdf}
        \caption{Logo UAMT}
        \label{obr:logo}
    \end{figure}

    % Ukázka tabulky
    \begin{table}[!h]
        \caption{Ukázka tabulky}
        \begin{center}
            \begin{tabular}{ |c|c|c| } 
                \hline
                Jméno & Příjmení & ID \\ 
                \hline\hline
                Jan   & Novák    & 16 \\ 
                \hline
                Petr  & Novák    & 23 \\ 
                \hline
            \end{tabular}
        \end{center}
        \label{tab:ukazka}
    \end{table}
    
    % Ukázka rovnice
    \begin{equation} \label{rov:pythagor}
        c^{2} = a^{2} + b^{2}
    \end{equation}
    
    % Ukázka číslovaného výčtu
    \begin{enumerate}
        \item Úkol č.1
        \item Úkol č.2
    \end{enumerate}
    
    % Ukázka výčtu
    \begin{itemize}
        \item Přístroj č.1
        \item Přístroj č.2
    \end{itemize}
    
    % Příklad využití odkazů v textu:
    Na obrázku \ref{obr:logo} se nachází \dots\\
    V tabulce \ref{tab:ukazka} je uveden \dots\\
    Rovnice \eqref{rov:pythagor} definuje vztah pro Pythagorovu větu.
    V kapitole \nameref{kap:zadani} \dots
    
\end{comment}

\end{document} % Konec dokumentu
