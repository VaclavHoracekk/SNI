\documentclass{protokol}
\usepackage{array}
\usepackage{tabularx}
\usepackage{environ}
%------------------- Zde vyplňte údaje -------------------------------
\autor{Václav Horáček}
\autorID{256296}
\autorr{Jan Holík}
\autorrID{256295}
\rocnik{3}
\merenodne{23.\,9.\,2025}
\nazev{Měření otáček}
\predmet{Snímače}
%=====================================================================
\begin{document}
\maketitle                  % Vygeneruje titulní stránku podle vyplněných údajů
%\tableofcontents\newpage   % Vygeneruje obsah
%------------------- Zde začíná samotný dokument ---------------------

% command pro psani promennych k rovnicim
\newenvironment{conditions}
  {\par\vspace{\abovedisplayskip}\noindent\begin{tabular}{>{$}l<{$} @{${}={}$} l}}
  {\end{tabular}\par\vspace{\belowdisplayskip}}

% Zadává
\bibliographystyle{IEEEtran}

% Uprava tabulek
\def\arraystretch{1.3}
\setlength{\headheight}{15pt}
\renewcommand{\sectionmark}[1]{\markboth{#1}{}}

\newcolumntype{C}{>{\centering\arraybackslash}X}

% FUNKCE PRO GENEROVANI TABULEK 
%   1. argument popisek
%   2. argument format
%   3. argument label
%   Do tela psat data a \hline
\NewEnviron{protocoltable}[3][Tabulka]{%
    \begin{table}[!h]
        \centering
        \caption{#1}\label{tab:#3}
        \vspace{0.3cm}
        \begin{tabularx}{\textwidth}{#2}
            \BODY
        \end{tabularx}
    \end{table}
}

% FUNKCE PRO TISK OBRAZKU
%  1. argument titulek
%  2. argument cesta napr. src\neco.png
%  3. argument scale - velikost rozsah 0.0-1.0
%  4. argument label
\newcommand{\printfigure}[4][Obrazek]{%
    \begin{figure}[!h]
        \centering
        \includegraphics[scale=#3]{#2}
        \caption{#1}
        \label{obr:#4}
    \end{figure}
}


\section{ZADÁNÍ}\label{kap:zadani}
    \begin{enumerate}
        \item   Změřte a vyneste do grafu závislost výstupního napětí tachodynama na otáčkách
                v rozsahu ±2000 ot/min. Určete pomocí MNČ konstantu K tachodynama a porovnejte ji s údaji výrobce (vypočítejte relativní odchylku). Určete linearitu. Nejistotu
                konstanty K určete ze dvou měřených bodů (pro tyto dva body hodnotu otáček
                změřte pomocí čítače).

        \item   Určete počet lamel komutátoru tachodynama.
                
        \item   U fotoelektrického odrazového snímače stanovte kolik impulzů připadá na jednu
                otáčku. Na čem závisí tato hodnota? Je možné na daném přípravku dosáhnout
                různých výsledků? Podmínky měření si zaznamenejte!
        
        \item   Na osciloskopu si prohlédněte a zaznamenejte tvar výstupních impulzů indukčního
                snímače a Hallovy sondy pro levé ozubené kolo. Průběh si zakreslete spolu s prů
                během vzdálenosti čela snímače od ozubeného kola tak, aby byla patrná souvislost
                výstupního signálu s tvarem ozubeného kola. Kdy se indukuje napětí na výstupu
                snímačů?

        \item   Zaznamenejte průběh signálů pro různé typy ozubených kol, včetně integrace. Jak
                souvisí tvar zubu a průběh integrálu výstupního napětí? U kterého tvaru zubu lze
                rozlišit směr otáčení?

        \item   Zobrazte na osciloskopu výstupní signál z optického inkrementální snímače a kvadraturního dekodéru pro oba směry otáčení. Průběhy si zaznamenejte (důležitá je
                fáze signálů) a zhodnoťte, jak se projeví změna směru na výstupních signálech. U
                kvadraturního dekodéru určete, v jakém módu pracuje (x1, x2 nebo x4). Srovnejte
                s teoretickými předpoklady.
                
        \item   Určete rozlišení inkrementálního optického snímače (počet impulzů na jednu otáčku)
                pomocí čítače.
                
        \item   Změřte efektivní hodnotu výstupních napětí resolveru v závislosti na úhlu natočení v rozsahu 0 až 360$^\circ$. Pro oba výstupy stanovte body, ve kterých se mění fáze
                vzhledem k budícímu signálu Uref . V intervalech vymezených těmito body změřte,
                má-li signál souhlasnou nebo opačnou fázi. Naměřená napětí vyneste do grafu. Fázi
                v grafu rozlište znaménkem (opačná = záporné). Z naměřených napětí vypočtěte
                úhel natočení a vyčíslete chybu v procentech z rozsahu. Změřte pracovní frekvenci
                resolveru (Uref ).

        \item   Na přípravku nastavte otáčky 900 ot/min, stroboskopem určete přesnou hodnotu a
                vypočítejte relativní odchylku.
    \end{enumerate}

\section{Úkol 1 - Převodní charakteristika tachodynama}
    hello \cite{navod}
    \begin{conditions}
        \alpha     &  notional permeability factor \\
        N     &  number of waves \\   
        S_{d} &  damage level
    \end{conditions}
    \subsection{Teoretický rozbor}
    \subsection{Použité přístroje a přípravky}
    \subsection{Postup měření}
    \subsection{Naměřené hodnoty}
    \subsection{Zpracované výsledky měření}
    \subsection{Závěr}

\pagebreak

\section{Úkol 2 - Lamely tachodynama}
    \subsection{Teoretický rozbor}
    \subsection{Použité přístroje a přípravky}
    \subsection{Postup měření}
    \subsection{Naměřené hodnoty}
    \subsection{Zpracované výsledky měření}
    \subsection{Závěr}

\pagebreak

\section{Úkol 3 - Odrazový snímač}
    hello
    \subsection{Teoretický rozbor}
    \subsection{Použité přístroje a přípravky}
    \subsection{Postup měření}
    \subsection{Naměřené hodnoty}
    \subsection{Zpracované výsledky měření}
    \subsection{Závěr}

\pagebreak

\section{Úkol 4 - Indukční snímač/Hallova sonda}
    \subsection{Teoretický rozbor}
    \subsection{Použité přístroje a přípravky}
    \subsection{Postup měření}
    \subsection{Naměřené hodnoty}
    \subsection{Zpracované výsledky měření}
    \subsection{Závěr}
\pagebreak

\section{Úkol 5 - Průběhy signálů z měření ozubených kol}
    \subsection{Teoretický rozbor}
    \subsection{Použité přístroje a přípravky}
    \subsection{Postup měření}
    \subsection{Naměřené hodnoty}
    \subsection{Zpracované výsledky měření}
    \subsection{Závěr}

\pagebreak


\section{Úkol 6 - Inkrementálního optický snímač/Kvadratutní dekodér}
    \subsection{Teoretický rozbor}
    \noindent Rozlišení závisí na počtu rysek optické clony. Skládá se ze zdroje světla, otočné clony ,tří fotocitlivých přijímačů a tvarovacích obvodů. Paprsek světla je přerušován značkami na cloně. Vznikají obdelníkové signály. Výstupní signály kanálů A a B jsou vzájemně fázově posunuty o 90°. Toto uspořádání se nazývá kvadraturní dekodér (slouží k vyhodnocení směru). Ten může pracovat s rozlišením x1(jedna hrana signálu), x2(náběžná i sestupná hrana) a x4(náběžná i sestupná hrana obou signálů). Při kladném směru otáčení je při náběžné hraně signálu A signál B vždy v logické úrovni 1. Při záporném směru otáčení je to naopak. 

    \subsection{Použité přístroje a přípravky}
    \subsection{Postup měření}
    \begin{enumerate}
        \item Připojili jsme na desku modul s inkrementálním optickým snímačem.
        \item Na kanály X a Z jsme postupně přivedli signály A,B,U,D. Na vstup EXT jsme přivedli Z.
        \item Zobrazili jsme a uložili výstupní signály z optického inkrementálního snímače a kvadraturního dekodéru.
    \end{enumerate}


    \clearpage
    \subsection{Naměřené hodnoty}
    \printfigure[Výstupy A a D při záporném směru otáčení]{src/6_ADMinus.png}{0.6}{macho}
    \printfigure[Výstupy A a D při kladném směru otáčení]{src/6_ADPlus.png}{0.6}{macho}
    \printfigure[Výstupy B a U při záporném směru otáčení]{src/6_BUMinus.png}{0.6}{macho}
    \printfigure[Výstupy B a U při při kladném směru otáčení]{src/6_BUPlus.png}{0.6}{macho}
    \printfigure[Výstupy A a B při při záporném směru otáčení]{src/6_zaporny_smer.png}{0.6}{macho}
    \printfigure[Výstupy A a B při při kladném směru otáčení]{src/6_kladny_smer.png}{0.6}{macho}

    \clearpage
    \subsection{Zpracované výsledky měření}
    \noindent Z obrázků je jasně vidět, že když se clona otáčí v kladném směru, tak čítač snižuje svou hodnotu. Pokud se clona otáči v záporné směru, hodnota čítače se zvyšuje. \\
    \noindent Na obrazcích je vidět, že za jednu periodu signálu A/B proběhnou 4 pulzy signálu U/D tzn., že kvadratutní dekodér reaguje na náběžnou i sestupnou hranu signálů A a B, které jsou vzájemně fázově posunuty o 90°. Z toho lze vyvodit že dekodér pracuje v módu x4.

    \subsection{Závěr}
    \noindent Zobrazili jsme na osciloskopu výstupní signály optického inkrementálního snímače a kavdraturního dekodéru pro oba směry otáčení. Z uložených obrázků je jasně poznat, že dekodér pracuje v módu x4.

\pagebreak


\section{Úkol 7 - Rozlišení inkrementálního optického snímače}
    \subsection{Teoretický rozbor}
    \noindent Rozlišení závisí na počtu rysek optické clony. Skládá se ze zdroje světla, otočné clony ,tří fotocitlivých přijímačů a tvarovacích obvodů. Paprsek světla je přerušován značkami na cloně. Vznikají obdelníkové signály. Výstupní signály kanálů A a B jsou vzájemně fázově posunuty o 90°. Toto uspořádání se nazývá kvadraturní dekodér (slouží k vyhodnocení směru). Ten může pracovat s rozlišením x1(jedna hrana signálu), x2(náběžná i sestupná hrana) a x4(náběžná i sestupná hrana obou signálů). Při kladném směru otáčení je při náběžné hraně signálu A signál B vždy v logické úrovni 1. Při záporném směru otáčení je to naopak. 

    \subsection{Použité přístroje a přípravky}

    \subsection{Postup měření}

    \begin{enumerate}
        \item Připojili jsme na desku modul s inkrementálním optickým snímačem.
        \item Změřili jsme počet pulzů na jednu otáčku pomocí čítače.
    \end{enumerate}

    \subsection{Naměřené hodnoty}

    \noindent Změřená hodnota: 2049,9 pulzů za otáčku

    \subsection{Zpracované výsledky měření}
        \begin{equation} \label{rov:pythagor}
        2049,9 \doteq 2048 \text{pulzů za otáčku}
    \end{equation} 
            

    \subsection{Závěr}
    \noindent Pomocí čítače jsme změřili počet pulzů za otáčku. Naměřená hodnota 2049.9 se téměř neliší od výrobcem udávané hodnoty 2048 pulzů za otáčku. Odchylka může být způsobena neideálností měřících přístrojů, které byly použity a vlivem okolí.


\pagebreak
\section{Úkol 8 - Výstupní napětí resolveru}
    \subsection{Teoretický rozbor}
    \noindent Používá se k měření absolutní hodnoty úhlové polohy. Dvě vzájemně pootočené statorové vinutí o 90° a jedno rotorové vinutí. Na vinutí rotoru přivedeme střídavý proud. Vzniká magnetické pole. To indukuje napětí v cívkách statoru jehož amplituda je dána úhlem natočení rotoru. V ustáleném stavu platí:

    \begin{equation} \label{rov:pythagor}
        \left( \dfrac{U_{\sin}}{U_{\cos}} \right) = \left(\dfrac{\sin{\varphi}}{\cos{\varphi}} \right) = \tan{\varphi}
    \end{equation} 






    \subsection{Použité přístroje a přípravky}
    \subsection{Postup měření}

    \begin{enumerate}
        \item Připojili jsme modul se stupnicí indikující natočení a nastavili nulovou polohu resolveru.
        \item Připojili jsme výstupy resolveru $U_{\sin}$ a $U_{\cos}$ na kanály X a Y osciloskopu a refernční signál $U_{ref}$ na vstup externí synchronizace. Změřili jsme efektivní hodnoty napětí $U_{\sin}$ a $U_{\cos}$ pro všechna natočení s krokem 15°.
        \item Následně jsme změřili fázi obou výstupních signálů resolveru vůči refernčnímu signálu.
        \item Z naměřených dat jsme vynesli do grafu průběh výstupních napětí resolveru a spočítali úhel natočení, který jsme následně porovnali s predpokládaným úhlem natočení.
    \end{enumerate}

    \clearpage
    \subsection{Naměřené hodnoty}

    \begin{protocoltable}[Závislost $U_{\sin}$ na úhlu natočení $\varphi_{ref}$]{|C|C|C|C|C|C|}{ukazka}

    \hline
    $\varphi_{ref}$[°]  & 0 & 15 & 30 & 45 & 60 \\
    \hline
    $U_{\sin}$[V] & -131.9 & 120.5 & 333.8 & 528.4 & 687.6   \\
    \hline
    $\varphi_{ref}$[°]  & 75 & 90 & 105 & 120 & 135 \\
    \hline
    $U_{\sin}$[V]  & 801.5 & 862.7 & 867.1 & 812.9 & 701.4   \\
    \hline
    $\varphi_{ref}$[°] & 150 & 165 & 180 & 195 & 210 \\
    \hline
    $U_{\sin}$[V] & 542.5 & 344.6 & 126.4 & -114.8 & -330.4    \\
    \hline
    $\varphi_{ref}$[°] & 225 & 240 & 255 & 270 & 285 \\
    \hline
    $U_{\sin}$[V] & -524.2 & -687.0 & -802.5 & -864.1 & -868.3 \\
    \hline
    $\varphi_{ref}$[°] & 300 & 315 & 330 & 345 & 360 \\
    \hline
    $U_{\sin}$[V] & -812.4 & -702.2 & -541.4 & -348.1 & -134.1  \\
    \hline
    \end{protocoltable}

    \begin{protocoltable}[Závislost $U_{\cos}$ na úhlu natočení $\varphi_{ref}$]{|C|C|C|C|C|C|}{ukazka}

    \hline
    $\varphi_{ref}$[°]  & 0 & 15 & 30 & 45 & 60 \\
    \hline
    $U_{\cos}$[mV] & 866.2 & 877.2 & 828.9 & 725.2 & 573.0 \\
    \hline
    $\varphi_{ref}$[°]  & 75 & 90 & 105 & 120 & 135 \\
    \hline
    $U_{\cos}$[V]  & 385.7 & 175.5 & -871.8 & -291.2 & -496.3  \\
    \hline
    $\varphi_{ref}$[°] & 150 & 165 & 180 & 195 & 210 \\
    \hline
    $U_{\cos}$[V] & -667.2 & -792.7 & -862.7 & -874.3 & -825.5    \\
    \hline
    $\varphi_{ref}$[°] & 225 & 240 & 255 & 270 & 285 \\
    \hline
    $U_{\cos}$[V] & -723.6 & -571.3 & -380.0 & -166.1 & 75.79 \\
    \hline
    $\varphi_{ref}$[°] & 300 & 315 & 330 & 345 & 360 \\
    \hline
    $U_{\cos}$[V] & 293.0 & 498.0 & 669.8 & 793.7 & 864.4  \\
    \hline
    \end{protocoltable}

    \clearpage
    \subsection{Zpracované výsledky měření}

    \printfigure[Závislost napětí na úhlu natočení]{src/Bez názvu.png}{0.65}{macho}

    \noindent Výpočet úhlu natočení:
    \begin{equation} \label{rov:pythagor}
        %c^{2} = a^{2} + b^{2}
        \varphi_{vyp} = \arctan \left( \dfrac{U_{\sin}}{U_{\cos}} \right) = \arctan \left(\dfrac{-131.9}{866.2} \right) = -8.658 \text{°}
    \end{equation} 

    \noindent Jelikož arctg nabývá pouze hodnot od -90° do +90°, tak logicky nejsme schopni při počítání s ním dosáhnou plného rozsahu 0-360°, proto jsme museli provést posun mezi kvadranty. \\

    \noindent Pro 1. kvadrant (tj. 0-90°) platí:
    \begin{equation} \label{rov:pythagor}
        \varphi_{prep} = \varphi_{vyp}
    \end{equation} 

    \noindent Pro 2. a 3. kvadrant (tj. 90-270°) platí:
    \begin{equation} \label{rov:pythagor}
        \varphi_{prep} = 180\text{°} + \varphi_{vyp}
    \end{equation}

    \noindent Pro 4. kvadrant (tj. 270-360°) platí:
    \begin{equation} \label{rov:pythagor}
        \varphi_{prep} = 360\text{°} + \varphi_{vyp}
    \end{equation}

    \noindent Výpočet absolutní odchylky:
    \begin{equation} \label{rov:pythagor}
        %c^{2} = a^{2} + b^{2}
        \Delta_{\varphi} = \varphi_{prep} - \varphi_{ref} = -8.658 - 0 = -8.658 \text{°}
    \end{equation}

     \begin{protocoltable}[Úhel natočení vypočtený z naměřených napětí a jeho absolutní odchylka]{|C|C|C|C|C|C|}{ukazka}

    \hline
     $\varphi_{ref}$[°]  & 0 & 15 & 30 & 45 & 60 \\
    \hline
    $U_{\sin}$[mV]  & -131.9 & 120.5 & 333.8 & 528.4 & 687.6 \\
    \hline
    $U_{\cos}$[mV] & 866.2 & 877.2 & 828.9 & 725.2 & 573.0 \\
    \hline
    $\varphi_{prep}$[°] & -8.658  & 7.822 & 21.93 & 36.08 & 50.19 \\
    \hline
    $\Delta_{\varphi}$[°] & -8.658 & -7.175 & -8.065 & -8.921 & -9.806\\
    \hline
    \hline

    $\varphi_{ref}$[°]  & 75 & 90 & 105 & 120 & 135 \\
    \hline
    $U_{\sin}$[mV]   & 801.5 & 862.7 & 867.1 & 812.9 & 701.4 \\
    \hline
    $U_{\cos}$[mV]  & 385.7 & 175.5 & -871.8 & -291.2 & -496.3  \\
    \hline
    $\varphi_{prep}$[°] & 64.30 & 78.50 & 95.74 & 109.7 & 125.3 \\
    \hline
    $\Delta_{\varphi}$[°] & -10.70 & -11.50 & -9.259 & -10.29 & -9.717 \\
    \hline
    \hline

    $\varphi_{ref}$[°]  & 150 & 165 & 180 & 195 & 210 \\
    \hline
    $U_{\sin}$[mV] & 542.5 & 344.6 & 126.4 & -114.8 & -330.4 \\
    \hline
    $U_{\cos}$[mV] & -667.2 & -792.7 & -862.7 & -874.3 & -825.5    \\
    \hline
    $\varphi_{prep}$[°] & 140.9 & 156.5 & 171.7 & 187.5 & 201.8 \\    
    \hline
    $\Delta_{\varphi}$[°] & -9.115 & -8.495 & -8.335 & -7.520 & -8.187 \\
    \hline
    \hline

    $\varphi_{ref}$[°]  & 225 & 240 & 255 & 270 & 285 \\
    \hline
    $U_{\sin}$[mV] & -524.2 & -687.0 & -802.5 & -864.1 & -868.3 \\
    \hline
    $U_{\cos}$[mV] & -723.6 & -571.3 & -380.0 & -166.1 & 75.79 \\
    \hline
    $\varphi_{prep}$[°] & 215.9 & 230.3 & 244.7 & 259.1 & 275.0 \\
    \hline
    $\Delta_{\varphi}$[°] & -9.079 & -9.746 & -10.34 & -10.88 & -10.01  \\
    \hline
    \hline
    
    $\varphi_{ref}$[°]  & 300 & 315 & 330 & 345 & 360 \\
    \hline
    $U_{\sin}$[mV] & -812.4 & -702.2 & -541.4 & -348.1 & -134.1 \\
    \hline
    $U_{\cos}$[mV] & 293.0 & 498.0 & 669.8 & 793.7 & 864.4  \\
    \hline
    $\varphi_{prep}$[°] & 289.8 & 305.3 & 321.1 & 336.2 & 351.2 \\
    \hline
    $\Delta_{\varphi}$[°] & -10.17 & -9.656 & -8.949 & -8.681 & -8.818 \\
    \hline
    \end{protocoltable}

    \newpage
    \subsection{Závěr}
    \noindent V tomto úkolu jsme měřili amplitudu výstupních napětí resolveru a jejich fázi. Při určování fáze u napětí jsme si nebyli jistí jaké znamínko zvolit v místech kde bylo napětí ve fázi. Zvolili jsme proto takové znamínko, které co nejlépe sedělo do harmonického průběhu. U výpočtu úhlu natočení nám všechny úhly vyšli posunuté a to v intervalu od -7,520° do -11,50°. Jelikož byl tento posun velmi konzistentní, tak usuzujeme, že se jedná o chybu aditivní. Také připouštíme, že mohlo dojít i k nepřesnostem při měření a vlivem vnějšího okolí.


\pagebreak


\section{Úkol 9 - Měření otáček stroboskopem}
    \subsection{Teoretický rozbor}
    \noindent Zařízení, které vytváří rychlé, periodické záblesky světla, aby vytvořilo iluzi zpomaleného nebo zmrazeného pohybu. Při souhlasném nastavení frekvence záblesků světla s rychlostí otáček se začne jevit tečka na otáčejícím se disku nehybně.

    \subsection{Použité přístroje a přípravky}
    \subsection{Postup měření}
    \begin{enumerate}
        \item Na otáčkoměru jsme nastavili otáčky.
        \item Pomocí stroboskopu jsme změřili rychlost otáček.
        
    \end{enumerate}



    \subsection{Naměřené hodnoty}

    $n_{ref}$ = 900 ot/min \\
    $n_{strob}$ = 900.5 ot/min

    \subsection{Zpracované výsledky měření}
    
    \begin{equation} \label{rov:pythagor}
        %c^{2} = a^{2} + b^{2}
        \delta_{n} = \dfrac{n_{ref}-n_{strob}}{n_{strob}}*100 = \dfrac{900-900.5}{900.5}*100 = -0.055 \%
    \end{equation}


    \subsection{Závěr}
    \noindent Nastavili jsme rychlost otáček na otáčkoměru a potom pomocí stroboskopu ověřili opravdovost hodnotu ukazovaných na otáčkoměru. Zjistili jsme, že se hodnota liší o -0,055\%. To je způsobeno vlivy okolí a neideálností měřících přístrojů.

\pagebreak

\section{Závěr}
\cite{navod}

\pagebreak

\bibliography{ref}

\pagebreak

%Ukazka tabulky a obrazku

\printfigure[Obrazek]{src/macho_emote.jpg}{0.5}{macho}



\begin{protocoltable}[Ukázková tabulka]{|C|C|C|C|C|C|}{ukazka}

    \hline
    $\alpha$[m] & 1.56 & 2 & 3 & 4 & 5 \\
    \hline
    $\varphi$[s] & 1 & 2 & 3 & 4 & 5\\
    \hline
\end{protocoltable}


%------------------------------ Ukázky -------------------------------
\begin{comment} % Komentář bloku
    \newpage % Začátek na nové stránce
        
    % Ukázka obrázku
    \begin{figure}[!h]
        \centering
        \includegraphics[scale=0.3]{UAMT_color_CMYK_CZ.pdf}
        \caption{Logo UAMT}
        \label{obr:logo}
    \end{figure}

    % Ukázka tabulky
    \begin{table}[!h]
        \caption{Ukázka tabulky}
        \begin{center}
            \begin{tabular}{ |c|c|c| } 
                \hline
                Jméno & Příjmení & ID \\ 
                \hline\hline
                Jan   & Novák    & 16 \\ 
                \hline
                Petr  & Novák    & 23 \\ 
                \hline
            \end{tabular}
        \end{center}
        \label{tab:ukazka}
    \end{table}
    
    % Ukázka rovnice
    \begin{equation} \label{rov:pythagor}
        c^{2} = a^{2} + b^{2}
    \end{equation}
    
    % Ukázka číslovaného výčtu
    \begin{enumerate}
        \item Úkol č.1
        \item Úkol č.2
    \end{enumerate}
    
    % Ukázka výčtu
    \begin{itemize}
        \item Přístroj č.1
        \item Přístroj č.2
    \end{itemize}
    
    % Příklad využití odkazů v textu:
    Na obrázku \ref{obr:logo} se nachází \dots\\
    V tabulce \ref{tab:ukazka} je uveden \dots\\
    Rovnice \eqref{rov:pythagor} definuje vztah pro Pythagorovu větu.
    V kapitole \nameref{kap:zadani} \dots
    
\end{comment}

\end{document} % Konec dokumentu
