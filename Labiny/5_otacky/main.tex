\documentclass{protokol}
\usepackage{array}
\usepackage{tabularx}
\usepackage{environ}
%------------------- Zde vyplňte údaje -------------------------------
\autor{Václav Horáček}
\autorID{256296}
\autorr{Jan Holík}
\autorrID{256295}
\rocnik{3}
\merenodne{23.\,9.\,2025}
\nazev{Měření otáček}
\predmet{Snímače}
%=====================================================================
\begin{document}
\maketitle                  % Vygeneruje titulní stránku podle vyplněných údajů
%\tableofcontents\newpage   % Vygeneruje obsah
%------------------- Zde začíná samotný dokument ---------------------

% command pro psani promennych k rovnicim
\newenvironment{conditions}
  {\par\vspace{\abovedisplayskip}\noindent\begin{tabular}{>{$}l<{$} @{${}-{}$} l}}
  {\end{tabular}\par\vspace{\belowdisplayskip}}

% Zadává
\bibliographystyle{IEEEtran}

% Uprava tabulek
\def\arraystretch{1.3}
\setlength{\headheight}{15pt}
\renewcommand{\sectionmark}[1]{\markboth{#1}{}}

\newcolumntype{C}{>{\centering\arraybackslash}X}

% FUNKCE PRO GENEROVANI TABULEK 
%   1. argument popisek
%   2. argument format
%   3. argument label
%   Do tela psat data a \hline
\NewEnviron{protocoltable}[3][Tabulka]{%
    \begin{table}[!h]
        \centering
        \caption{#1}\label{tab:#3}
        \vspace{0.3cm}
        \begin{tabularx}{\textwidth}{#2}
            \BODY
        \end{tabularx}
    \end{table}
}

% FUNKCE PRO TISK OBRAZKU
%  1. argument titulek
%  2. argument cesta napr. src\neco.png
%  3. argument scale - velikost rozsah 0.0-1.0
%  4. argument label
\newcommand{\printfigure}[4][Obrazek]{%
    \begin{figure}[!h]
        \centering
        \includegraphics[scale=#3]{#2}
        \caption{#1}
        \label{obr:#4}
    \end{figure}
}


\section{ZADÁNÍ}\label{kap:zadani}
    \begin{enumerate}
        \item   Změřte a vyneste do grafu závislost výstupního napětí tachodynama na otáčkách
                v rozsahu ±2000 ot/min. Určete pomocí MNČ konstantu K tachodynama a porov-
                nejte ji s údaji výrobce (vypočítejte relativní odchylku). Určete linearitu. Nejistotu
                konstanty K určete ze dvou měřených bodů (pro tyto dva body hodnotu otáček
                změřte pomocí čítače).

        \item   Určete počet lamel komutátoru tachodynama.
                
        \item   U fotoelektrického odrazového snímače stanovte kolik impulzů připadá na jednu
                otáčku. Na čem závisí tato hodnota? Je možné na daném přípravku dosáhnout
                různých výsledků? Podmínky měření si zaznamenejte!
        
        \item   Na osciloskopu si prohlédněte a zaznamenejte tvar výstupních impulzů indukčního
                snímače a Hallovy sondy pro levé ozubené kolo. Průběh si zakreslete spolu s prů
                během vzdálenosti čela snímače od ozubeného kola tak, aby byla patrná souvislost
                výstupního signálu s tvarem ozubeného kola. Kdy se indukuje napětí na výstupu
                snímačů?

        \item   Zaznamenejte průběh signálů pro různé typy ozubených kol, včetně integrace. Jak
                souvisí tvar zubu a průběh integrálu výstupního napětí? U kterého tvaru zubu lze
                rozlišit směr otáčení?

        \item   Zobrazte na osciloskopu výstupní signál z optického inkrementální snímače a kva-
                draturního dekodéru pro oba směry otáčení. Průběhy si zaznamenejte (důležitá je
                fáze signálů) a zhodnoťte, jak se projeví změna směru na výstupních signálech. U
                kvadraturního dekodéru určete, v jakém módu pracuje (x1, x2 nebo x4). Srovnejte
                s teoretickými předpoklady.
                
        \item   Určete rozlišení inkrementálního optického snímače (počet impulzů na jednu otáčku)
                pomocí čítače.
                
        \item   Změřte efektivní hodnotu výstupních napětí resolveru v závislosti na úhlu nato-
                čení v rozsahu 0 až 360$^\circ$. Pro oba výstupy stanovte body, ve kterých se mění fáze
                vzhledem k budícímu signálu Uref . V intervalech vymezených těmito body změřte,
                má-li signál souhlasnou nebo opačnou fázi. Naměřená napětí vyneste do grafu. Fázi
                v grafu rozlište znaménkem (opačná = záporné). Z naměřených napětí vypočtěte
                úhel natočení a vyčíslete chybu v procentech z rozsahu. Změřte pracovní frekvenci
                resolveru (Uref ).

        \item   Na přípravku nastavte otáčky 900 ot/min, stroboskopem určete přesnou hodnotu a
                vypočítejte relativní odchylku.
    \end{enumerate}

\section{Úkol 1 - Převodní charakteristika tachodynama}
    Cílem měření bylo změřit závislost výstupního napětí na měřených otáčkách, určit převodní konstantu $K$ achybu linearity. Je nutné poznamenat, 
    že pro tento bod není vypočtena nejistota kvůli organizačním pokynům k prvnímu měření. 
    \subsection{Teoretický rozbor}
    Tachodynamo je DC motor, který místo běžného pracovního režimu, kdy se transformuje elektrický proud na kroutící moment, koná tento převod naopak.
    Otáčením hřídele tachodynama je generováno stejnosměrné elektrické napětí $U$, které je přímo úměrné otáčkám. Převodní konstantu je možné
    odečíst jako směrnici převodní charakteristiky $U = f(n)$. Chyba linearity průběhu bývá menší jak 1$\%$\cite{navod}.

    Pomocí tachodynama lze měřit velice jednoduše díky vysoké hodnotě výstupního signálu a možnosti zjištění směru otáčení díky polaritě výstupního signálu. 
    
    Nevýhodou je vysoká zvlněnost výstupního signálu a nutnost senzor zatěžovat minimálně. 

    Napětí na svorkách tachodynama lze vyjádřit následujícím způsobem:
    \begin{equation}
        E = B \cdot l \cdot v \rightarrow U = \dfrac{B \cdot l \cdot \pi \cdot d \cdot n}{60} \text{\quad [V; T; m; m; ot/min]}
    \end{equation}  
    
    \begin{conditions}
        E & elektromotorické napětí [V] \\
        B & magnetické sycení [T] \\
        l & aktivní délka vodiče [m] \\
        U & elektrické napětí [V] \\
        d & průměr vinutí [m] \\
        n & otáčky [ot/min] 
    \end{conditions}  
    
    \subsection{Postup měření}
        \begin{enumerate}
            \item Na základní desku s motorem bylo připojen přípravek s tachodynamem.
            \item K výstupním svorkám byl připojen multimetr Keysight 34450A (DC napětí).
            \item Pro rozsah otáček $[-2000, 2000]$ ot/min po kroku 250 ot/min odečítáno měřené DC napětí na svorkách tachodynama.
            \item Z naměřených hodnot byla vypočtena převodní konstanta $K$ a chyba linearity $\delta_n$.
        \end{enumerate}
    \pagebreak
    \subsection{Naměřené hodnoty}
        \begin{protocoltable}[Měřená napětí $U$ na svorchách tachodynama v závislosti na otáčkách $n$ ]{|C|C|C|C|C|C|}{tachodynamo}
            \hline
            $n$[ot/min] & -2000 & -1750 & -1500 & -1250 & -1000 \\
            \hline
            $U$[V] & -3.9757 & -3.4787 & -2.9858 & -2.4873 & -1.9902\\
            \hline
            $n$[ot/min] & -750 & -500 & -250 & -0 & 250 \\
            \hline
            $U$[V] & -1.4922 & -0.99557 & -0.50150 & -0.000092 & 0.50022 \\
            \hline
            $n$[ot/min] & 500 &  750 & 1000 & 1250 & 1500 \\
            \hline
            $U$[V] & 1.0033 & 1.5076 & 2.0117 & 2.5144 & 2.9930\\
            \hline
            $n$[ot/min] & 1500 & 2000 & X & X & X \\
            \hline
            $U$[V] & 3.4802 & 3.9793 & X & X & X\\
            \hline
        \end{protocoltable}

    \subsection{Zpracované výsledky měření}
        Pomocí metody nejmenších čtverců\cite{vypocty}(0.12) byla určena lineární regrese naměřených bodů:
        \begin{equation}
            K = \left( \dfrac{m \sum_{i}^{m} U_i n_i - \sum_{i}^{m} U_i \sum_{i}^{m} n_i }{n \sum_{i}^{m} n_i^2 - \left(\sum_{i}^{m} n_i \right)^2} \right) = \dfrac{17 \cdot 50810 + 0}{25500000 + 0} = 0.001993 \text{ V·min·ot$^{-1}$}
        \end{equation}
        \begin{equation}
            U_0 = \dfrac{1}{n} \left( \sum_{i}^{n} U_i - K \sum_{u}^{n} n_i\right) = \dfrac{1}{17} \left( 0.08265 - 0.001993 \cdot 0 \right) = 0.08265 \text{ V}
        \end{equation}

        Lineární aproximace převodní charakteristiky je tedy:
        \begin{equation}
            U_L = K\cdot n + u_0 = 0.001993 \cdot n + 0.08265 \text[ V]
        \end{equation}

        Výrobce udává pro měřené tachodynamo hodnotu K = 2 V/1000 ot·min$^{-1}$\cite{navod}. Relativní odchylka $\delta_K$ je tedy:
        \begin{equation}
            \delta_K = \dfrac{K_{m} - K_v}{K_v} \cdot 100 = \dfrac{0.001993 - 0.002}{0.002} \cdot 100  = -0.35\%
        \end{equation}

        Výpočet chyby linearity\cite{vypocty}(0.10):
        \begin{equation}
            \delta_L = \left( \left| \dfrac{U_N - U_L}{U_{Lmax} - U_{Lmin}} \right| \right)_{max} \cdot 100  = \dfrac{2.514 - 2.491}{3.985 - (-3.985)} \cdot 100 = 0.2975\%
        \end{equation}
        \linebreak
        \printfigure[Převodní charakteristika tachodynama]{src/graf1.png}{0.32}{tachodynamo}
    \subsection{Závěr}
        Pro změřená napětí vychází převodní charakteristika jako lineární s nejvyšší chybou linearity pouze 0.2975$\%$, což je i o proti obvyklým nízkým hodnotám velice dobrý výsledek.
        
        Oproti katalogové hodnotě vyšla relativní odchylka převodní konstanty \linebreak $\delta_K$ = -0.35$\%$. 

        Nejistota kostanty $K$ kvůli pokynům k prvním laboratorním cvičením nebyla vypočtena.

\pagebreak

\section{Úkol 2 - Počet lamel tachodynama}
    Cílem měření je pomocí zvlnění průběhu výstupního napětí na nízkých otáčkách určit počet lamel tachodynama. 
    \subsection{Teoretický rozbor}
        V kartáčových DC motorech se polarita budícího napětí mění za běhu pomocí lamel, kterých se musí kotva rotoru dotýkat a tím měnit polaritu magnetického pole k otáčení.
        
        Už jen to, že se kotva rotoru nedotýká pořád, způsobuje zvlnění výstupního napětí. Podle zvlnění na průběhu zvlnění lze určit spočítáním period zvlnění na otáčku tachodynama.
    \subsection{Postup měření}
        \begin{enumerate}
             \item Na základní desku s motorem bylo připojen přípravek s tachodynamem.
             \item Výstup tachodynama a výstup z odrazového snímače byl vyveden na osciloskop Siglent SDS 1102X+.
             \item Za nízkých otáček při AC vazbě vstupu byl změřen průběh zvlnění výstupního napětí.
             \item Ze zobrazeného průběhu byl spočítán počet lamel.
        \end{enumerate}
    \subsection{Naměřené hodnoty}
        \printfigure[Měřené zvlnění výstupního napětí tachodynama]{src/2_lamely.png}{0.55}{lamely}

    \pagebreak
    \subsection{Zpracované výsledky měření}
        Aby bylo možné spočítat počet lamel, tak je nutné si určit délku jedné otáčky a k ní počet period "harmonického" zvlnění vztáhnout.

        Na druhý vstup osciloskopu byl přiveden výstup z odrazového snímače. Čtyři periody tohoto snímače jsou jedna otáčka motoru a teď už jen stačí spočítat počet period harmonického zvlnění, kterých se v průběhu na jednu otáčku vyskytuje 19.

   
    \subsection{Závěr}
        Naměřený počet lamel je rovný s reálným počtem lamel na měřeném tachodynamu. V obou případech vychází 19 lamel.

\pagebreak

\section{Úkol 3 - Odrazový snímač}
    Cílem měření bylo zjisti, kolik impulzů připadá na jednu otázku, na čem tato hodnota závisí a zda-li lze na přípravku změnit.
    \subsection{Teoretický rozbor}
        Odrazový snímač je LED a prvek reagující na světlo(např. fototranzistor)\cite{navod}. V reakci na odraženého signálu z led diody od otáčející se hřídele generuje snímač obdélníkové pulzy.

        Frekvence pulzů je závislá na otáčkách hřídele a na počtu odrazových ploch \cite{navod}.

        \printfigure[Schématický náčrtek fotoelektrického reflexního snímače\cite{navod}]{src/fotoelektro.png}{0.9}{fotoelektro}
    \subsection{Postup měření}
        \begin{enumerate}
            \item Signál z odrazového snímače byl připojen na osciloskop Siglent SDS 1102X+.
            \item Ze spočítání odrazových ploch na motoru a přípravku s tachodynamem byl stanoven počet pulzů na otáčku.
            \item Po odpojení přípravku byly znovu spočítány odrazové plochy.
            \item Byly zapsány podmínky měření($t = 24.4 ^\circ$C, $P = 988.1$hPa, $RV = 58.1 \%$)
        \end{enumerate}
    \subsection{Naměřené hodnoty}
        \begin{itemize}
            \item Počet odrazových ploch (obdélníkových pulzů na otáčku) s připojeným přípavkem: 4
            \item -//- bez připojeného přípavku: 2
        \end{itemize}
    \subsection{Závěr}
        Pro změnu měřené frekvence obdélníkových pulzů, které jsou výstupem odrazového snímače je možné na přípravku buď možné změnit otáčky motoru nebo odpojit přípravek s jiným snímačem.

        Pokud je připojen přípravek tak jsou pro snímač připraveny 4 odrazové plochy, tudíž by se pro získání frekvence otáčení musela frekvence výstupní pulzů dělit 4.

        V případě odpojeného přípravku je počet odrazových ploch tudíž by se pro získání frekvence otáčení dělit pouze 2.

\pagebreak

\section{Úkol 4 - Indukční snímač/Hallova sonda}
    V tomto úkolu bylo cílem zaznamenat průběhy signálů z indukčního snímače a snímače s Hallovou sondou pro levé ozubené kolo. 
    Zároveň bylo třeba vypozororovat souvislost tvaru ozubeného kola s výstupním signálem
    \subsection{Teoretický rozbor}
    
    Indukční snímač k detekci vodivých předmětů patří k jednomu z nejpouřívanějších na světě díky svému vysoké výstupní úrovni a jednoduchosti. Snímač využívá indukčního zákona pro elektromotrické napětí\cite{navod}:

    \begin{equation}
        E = -\dfrac{d\phi}{dt} \cdot N \text{\quad [V; Wb/s, -]}     
    \end{equation}
    \begin{conditions}
        E & elektromotorické napětí [V] \\
        \phi & magnetický tok [Wb/s] \\
        N & počet závitů cívky [-] 
    \end{conditions}

    Jelikož je snímač většinou používán k měření polohy, tak je jeho použití limitováno nízkými otáčkami a nutností použít ozubená kola nebo nepravdelné hřídele.

    Další snímač pro detekci otáčení ozubených kol je snímač s Hallovou sondou. Podobně jako indukční snímač, tak i snímač s Hallovou sondou reaguje výstupním dvoustavovým signálem na změnu magnetického pole.
    Tyto snímače mají automatické nastavování rozhodovací úrovně a omezený dolní mezní kmitočet. Tyto okolnosti mohou zabránit zjištěním, zda-li je snímač měří zub kola nebo mezeru\cite{navod}.
    \subsection{Postup měření}
        \begin{enumerate}
            \item Byl připojen přípravek se snímači a ozubenými koly.
            \item Výstupní signály byly přivedeny na vstup osciloskopu Siglent SDS1102+.
            \item Průběhy signálů byly zaznamenány pomocí PC interface pro osciloskop.
        \end{enumerate}
        
    \pagebreak
    \subsection{Naměřené hodnoty}
        \printfigure[Průběhy výstupních signálů z indukčního a snímače s Hallovou sondou]{src/4_hall_indukce.png}{0.55}{indukce_hall}

    \subsection{Závěr}
        Ze zobrazených průběhů lze vypozorovat, že jsou snímače od sebe umístěny tak, že ve chvíli kdy na indukční zachytí zub, tak snímač s Hallovou sondou detekuje mezeru.

        Toto chování si lze odvodit z fyzikální podstaty indukčního snímače. Pokud si určíme signál ze senzoru jako kopii reálného otáčení kola, kdy je vysoká úroveň zub a nízká mezera, tak jeho derivace a otočení polarity vytvoří signál podobný výstupu indukčnímu snímači, což je ve shodě s rovnicí (7).

        Z toho plyne i to, že napětí se na snímači s Hallovou sondou generuje úměrně magnetickému poli, které je nejvyšší při přítomnosti zubu u snímače. Indukční snímač generuje napětí při změně magnetického pole, což je v našem případě přechod zub/mezera.

\pagebreak

\section{Úkol 5 - Průběhy signálů z měření ozubených kol}
    Cílem tohoto úkolu bylo zaznamenat výstupních průběhů indukčního snímače pro různé tvary ozubených kol a zjistit pro jaký tvar lze zjistit tvar otáčení kola.
    \subsection{Teoretický rozbor}
        V úkolu 4 bylo možné pozorovat derivační vztah mezi přítomností zubu u snímače a napětím na výstupu snímače, kdy bylo možné vidět jak hrantatý zub generuje špičatý napěťový pulz.

        Pokud bude zákon fungovat obdobně i pro jiné tvary zubů, tak je možné to pozorovat derivační charakter senzoru detailněji. Pomocí integrace je možné z průběhu výstupného napětí snímače odhadnou tvar zubů ozubeného kola a v některých případech i směr otáčení kola.
    \subsection{Postup měření}
        \begin{enumerate}
            \item Byl připojen přípravek se snímači a ozubenými koly.
            \item Výstupní signály byly přivedeny na vstup osciloskopu Siglent SDS1102+.
            \item Přípravek s motorem byl spuštěn na frekvenci otáčení v nižších stovkách ot/min.
            \item Na při osciloskopu byla zobrazen integrál napěťového výstupu snímače.
            \item Body 3 a 4 bylz provedeny pro všechny 4 ozubená kola.
            \item Průběhy signálů byly zaznamenány pomocí PC interface pro osciloskop.
        \end{enumerate}

    
    \subsection{Naměřené hodnoty}
        \printfigure[Průběh výstupního signálu pro kolo s hranatými zuby]{src/5_integral_1_kolo.png}{0.43}{hrany}
        \pagebreak
        \printfigure[Průběh výstupního signálu pro kolo s mělkými hranatými zuby]{src/5_integral_2_kolo.png}{0.43}{melke_hrany}
        \printfigure[Průběh výstupního signálu pro kolo s zuby ve tvaru rampy]{src/5_integral_3_kolo.png}{0.43}{rampa}
        \printfigure[Průběh výstupního signálu pro kolo s trojúhelníkovými zuby]{src/5_integral_4_kolo.png}{0.43}{trojuhelnik}
    \pagebreak
    \subsection{Závěr}
        Tvar integrálu výstupního napětí přibližně odpovídal tvaru zubů na odpovídajcím ozubeném kolu. Zajímavý bylo zejména třetí kolo, které mělo zuby ve tvaru pily (periodicky opakující-se lineární funkce).
        
        U tohoto kola je totiž možné pozorovat, kterým směrem se otáčí pomocí směrnice lineární části integrálu průběhu. Na obdélníkových a trojúhelníkových kolech nelze kvůli symetrii zubů pozorovat směr otáčení.

\pagebreak


\section{Úkol 6 - Inkrementálního optický snímač/Kvadratutní dekodér}
    \subsection{Teoretický rozbor}
    \subsection{Použité přístroje a přípravky}
    \subsection{Postup měření}
    \subsection{Naměřené hodnoty}
    \subsection{Zpracované výsledky měření}
    \subsection{Závěr}

\pagebreak


\section{Úkol 7 - Rozlišení inkrementálního optického snímače}
    \subsection{Teoretický rozbor}
    \subsection{Použité přístroje a přípravky}
    \subsection{Postup měření}
    \subsection{Naměřené hodnoty}
    \subsection{Zpracované výsledky měření}
    \subsection{Závěr}

\pagebreak
\section{Úkol 8 - Výstupní napětí resolveru}
    \subsection{Teoretický rozbor}
    \subsection{Použité přístroje a přípravky}
    \subsection{Postup měření}
    \subsection{Naměřené hodnoty}
    \subsection{Zpracované výsledky měření}
    \subsection{Závěr}

\pagebreak


\section{Úkol 9 - Měření otáček stroboskopem}
    \subsection{Teoretický rozbor}
    \subsection{Použité přístroje a přípravky}
    \subsection{Postup měření}
    \subsection{Naměřené hodnoty}
    \subsection{Zpracované výsledky měření}
    \subsection{Závěr}

\pagebreak

\section{Závěr}
    Při měření otáček tachodynamem bylo možné pozorovat lineární převodní charakteristiku s velice nízkou chybou linearity ($\delta_L = 0.2975\% $) a nízkou chybou převodní konstanty K = {$\delta_K$ = -0.35$\%$}.

\pagebreak

\section{Seznam použitých přístrojů}

\begin{protocoltable}[Seznam použitých přístrojů]{|C|C|C|}{pristroje}
    \hline
    Přístroj & Typ & Inventární číslo  \\
    \hline
    Keysight 34450A & Digitální multimetr & X \\
    \hline
    Hewlett Packard 53131A & Digitální čítač  & X \\
    \hline
    Siglent SDS 1102X+ & Digitální osciloskop  & X \\
    \hline
    Stroboskop DT-2249 & Stroboskop  & X \\
    \hline
    Přípravky se snímači & Přípravky  & X \\
    \hline
\end{protocoltable}

\pagebreak

\bibliography{ref}

\pagebreak

%Ukazka tabulky a obrazku







%------------------------------ Ukázky -------------------------------
\begin{comment} % Komentář bloku
    \newpage % Začátek na nové stránce
        
    % Ukázka obrázku
    \begin{figure}[!h]
        \centering
        \includegraphics[scale=0.3]{UAMT_color_CMYK_CZ.pdf}
        \caption{Logo UAMT}
        \label{obr:logo}
    \end{figure}

    % Ukázka tabulky
    \begin{table}[!h]
        \caption{Ukázka tabulky}
        \begin{center}
            \begin{tabular}{ |c|c|c| } 
                \hline
                Jméno & Příjmení & ID \\ 
                \hline\hline
                Jan   & Novák    & 16 \\ 
                \hline
                Petr  & Novák    & 23 \\ 
                \hline
            \end{tabular}
        \end{center}
        \label{tab:ukazka}
    \end{table}
    
    % Ukázka rovnice
    \begin{equation} \label{rov:pythagor}
        c^{2} = a^{2} + b^{2}
    \end{equation}
    
    % Ukázka číslovaného výčtu
    \begin{enumerate}
        \item Úkol č.1
        \item Úkol č.2
    \end{enumerate}
    
    % Ukázka výčtu
    \begin{itemize}
        \item Přístroj č.1
        \item Přístroj č.2
    \end{itemize}
    
    % Příklad využití odkazů v textu:
    Na obrázku \ref{obr:logo} se nachází \dots\\
    V tabulce \ref{tab:ukazka} je uveden \dots\\
    Rovnice \eqref{rov:pythagor} definuje vztah pro Pythagorovu větu.
    V kapitole \nameref{kap:zadani} \dots
    
\end{comment}

\end{document} % Konec dokumentu
