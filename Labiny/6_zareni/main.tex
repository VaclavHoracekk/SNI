\documentclass{protokol}
%------------------- Zde vyplňte údaje -------------------------------
\autor{Václav Horáček}
\autorID{123456}
\autorr{Petr Novotný}
\autorrID{654321}
\rocnik{2}
\merenodne{1.\,1.\,2022}
\nazev{Název protokolu}
\predmet{Prostředky průmyslové automatizace}
%=====================================================================
\begin{document}
\maketitle                  % Vygeneruje titulní stránku podle vyplněných údajů
%\tableofcontents\newpage   % Vygeneruje obsah
%------------------- Zde začíná samotný dokument ---------------------

\section{ZADÁNÍ}\label{kap:zadani}
Zde vložte zadání

\section{DOMÁCÍ PŘÍPRAVA}
Zde vložte obrázek domácí přípravy

\section{SEZNAM POUŽITÝCH PŘÍSTROJŮ}
Tuto kapitolu použijte s ohledem na aktuální situaci

\section{ZPRACOVANÉ ÚKOLY}
Zde zpracujte protokol

\section{ZÁVĚR}
Napište stručný závěr

%------------------------------ Ukázky -------------------------------
\begin{comment} % Komentář bloku
    \newpage % Začátek na nové stránce
        
    % Ukázka obrázku
    \begin{figure}[!h]
        \centering
        \includegraphics[scale=0.3]{UAMT_color_CMYK_CZ.pdf}
        \caption{Logo UAMT}
        \label{obr:logo}
    \end{figure}

    % Ukázka tabulky
    \begin{table}[!h]
        \caption{Ukázka tabulky}
        \begin{center}
            \begin{tabular}{ |c|c|c| } 
                \hline
                Jméno & Příjmení & ID \\ 
                \hline\hline
                Jan   & Novák    & 16 \\ 
                \hline
                Petr  & Novák    & 23 \\ 
                \hline
            \end{tabular}
        \end{center}
        \label{tab:ukazka}
    \end{table}
    
    % Ukázka rovnice
    \begin{equation} \label{rov:pythagor}
        c^{2} = a^{2} + b^{2}
    \end{equation}
    
    % Ukázka číslovaného výčtu
    \begin{enumerate}
        \item Úkol č.1
        \item Úkol č.2
    \end{enumerate}
    
    % Ukázka výčtu
    \begin{itemize}
        \item Přístroj č.1
        \item Přístroj č.2
    \end{itemize}
    
    % Příklad využití odkazů v textu:
    Na obrázku \ref{obr:logo} se nachází \dots\\
    V tabulce \ref{tab:ukazka} je uveden \dots\\
    Rovnice \eqref{rov:pythagor} definuje vztah pro Pythagorovu větu.
    V kapitole \nameref{kap:zadani} \dots
    
\end{comment}

\end{document} % Konec dokumentu
