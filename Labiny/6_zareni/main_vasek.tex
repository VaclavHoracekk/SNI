\documentclass[fleqn]{protokol}
\usepackage{array}
\usepackage{tabularx}
\usepackage{environ}

%------------------- Zde vyplňte údaje -------------------------------
\autor{Václav Horáček}
\autorID{256296}
\autorr{Jan Holík}
\autorrID{256295}
\rocnik{3}
\merenodne{14.\,10.\,2025}
\nazev{Měření ionizujícího záření}
\predmet{Snímače}
\teplota{21.4}
\tlak{992.0}
\vlhkost{37.3}
%=====================================================================

\newcommand{\neweq}{\\[0.8ex]}

\begin{document}

\maketitle                  % Vygeneruje titulní stránku podle vyplněných údajů
\tableofcontents\newpage   % Vygeneruje obsah
%------------------- Zde začíná samotný dokument ---------------------

% command pro psani promennych k rovnicim
\newenvironment{conditions}
  {\par\vspace{\abovedisplayskip}\noindent\begin{tabular}{>{$}l<{$} @{${}-{}$} l}}
  {\end{tabular}\par\vspace{\belowdisplayskip}}

% Zadává
\bibliographystyle{acm}

% Uprava tabulek
\def\arraystretch{1.3}
\setlength{\headheight}{15pt}
\renewcommand{\sectionmark}[1]{\markboth{#1}{}}

\newcolumntype{C}{>{\centering\arraybackslash}X}

% FUNKCE PRO GENEROVANI TABULEK 
%   1. argument popisek
%   2. argument format
%   3. argument label
%   Do tela psat data a \hline
\NewEnviron{protocoltable}[3][Tabulka]{%
    \begin{table}[!h]
        \centering
        \caption{#1}\label{tab:#3}
        \vspace{0.3cm}
        \begin{tabularx}{\textwidth}{#2}
            \BODY
        \end{tabularx}
    \end{table}
}

% FUNKCE PRO TISK OBRAZKU
%  1. argument titulek
%  2. argument cesta napr. src\neco.png
%  3. argument scale - velikost rozsah 0.0-1.0
%  4. argument label
\newcommand{\printfigure}[4][Obrazek]{%
    \begin{figure}[!h]
        \centering
        \includegraphics[scale=#3]{#2}
        \caption{#1}
        \label{obr:#4}
    \end{figure}
}


\section{Zadání}\label{kap:zadani}
    \begin{enumerate}
        \item Body zadání 1.-6. realizujte s $\beta$-zářiči Sr-90. Proveďte základní dozimetrická měření:
        
            \begin{enumerate}
                \item Změřte hodnotu přirozeného pozadí ionizujícího záření pomoci dozimetru Gamma-Scout.
                
                \item Zvýšené hodnoty záření způsobené $\beta$-zářičem během laboratorního cvičení pomocí dozimetru Gamma-Scout.
            
                \item V tabulce přehledně porovnejte hodnoty naměřené dozimetrem s hodnotami odpovídajícími hygienickým limitům.
            \end{enumerate}
           
        \item Proměřte impulzovou charakteristiku GM trubice, určete délku „plateau“ \linebreak a jeho strmost.

        \item Určete mrtvou dobu GM trubice, napájecí napětí volte ze středu plateau, porovnejte výsledky získané metodou dvou zářičů a přímým měřením na osciloskopu.

        \item Určete vliv stínící přepážky při měření závislosti $I = f(d)$, kde $d$\linebreak je tloušťka materiálu. Závislost vyneste do grafu. Určete součinitel zeslabení $\mu$  \linebreak a polotloušťku $d_{1/2}$.

        \item Určete hmotnostní koeficient útlumu $\mu_m$ charakterizující útlum ionizujícího záření v předložených materiálech, porovnejte hodnotu získanou výpočtem pro každý materiál s hodnotou stanovenou z grafu závislosti funkce intenzity \linebreak plošné hustotě.

        \item Stanovte nejistotu určení hustoty pro jeden materiál.

        \item S $\gamma$-zářičem Cs-137 změřte četnost impulzů pro sadu vzorků s různou hustotou \linebreak a porovnejte závislost počtu pulzů na hustotě s měřením s $\beta$-zářičem \linebreak z bodu 5.

    \end{enumerate}

\pagebreak 

% Ukol 1 - 
\section{Úkol 1 - Základní dozimetrická měření}
    \subsection{Teoretický rozbor}
    \subsection{Postup měření}
    \subsection{Naměřené hodnoty}   
    \subsection{Zpracované výsledky měření}
    \subsection{Závěr}

\pagebreak

% Ukol 2 - 
\section{Úkol 2 - Impulzová charakteristika GM trubice}
    \subsection{Teoretický rozbor}
    \subsection{Postup měření}
    \subsection{Naměřené hodnoty}   
    \subsection{Zpracované výsledky měření}
    \subsection{Závěr}

\pagebreak

% Ukol 3 - 
\section{Úkol 3 - Mrtvá doba GM trubice}
    \subsection{Teoretický rozbor}
    \subsection{Postup měření}
    \subsection{Naměřené hodnoty}   
    \subsection{Zpracované výsledky měření}
    \subsection{Závěr}

\pagebreak

% Ukol 4 - 
\section{Úkol 4 - Vliv stínící přepážky}
    \subsection{Teoretický rozbor}
    \subsection{Postup měření}
    \subsection{Naměřené hodnoty}   
    \subsection{Zpracované výsledky měření}
    \subsection{Závěr}

\pagebreak

% Ukol 5 - 
\section{Úkol 5 - Hmotnostní koeficient útlumu}
    \subsection{Teoretický rozbor}

        \begin{equation}
            I = I_0 \cdot e^{-\mu \cdot h} \text{\quad[m$^{-2}$ $\cdot$ s$^{-1}$]}
        \end{equation}         

        \begin{equation}
            V = a \cdot b \cdot h \text{\quad[m; m; m; m$^3$]}
        \end{equation}

        \begin{equation}
            \rho = \dfrac{m}{V} \text{\quad[kg; m$^3$; kg$\cdot$m$^{-3}$]}
        \end{equation}

        \begin{equation}
            \sigma = \rho \cdot h \text{\quad[ kg$\cdot$m$^{-3}$; m; kg$\cdot$m$^{-2}$]}
        \end{equation}


    \subsection{Postup měření}

    \subsection{Naměřené hodnoty}  
        \begin{protocoltable}[Naměřené rozměry, hmotnosti a počty pulzů za minutu pro uvedené vzorky]{|C|C|C|C|C|C|}{ukol4-mereni}
            \hline
            Vzorek & $a$[mm] & $b$[mm] & $h$[mm] & $m$[g] & $n$[min$^{-1}$] \\
            \hline
            1 &  128.6 & 98.10 & 0.2000 & 3.130 & 19560 \\
            \hline
            2 & 149.9 & 100.0 & 2.500 & 49.52 & 1840 \\
            \hline
            3 & 119.4 & 99.90 & 1.000 & 20.00 & 7139 \\
            \hline
            4 & 150.4 & 100.3 & 1.150 & 39.98 & 3442 \\
            \hline
            5 & 101.2 & 100.4 & 3.500 & 34.36 & 1801 \\
            \hline
            6 & 98.60 & 101.1 & 11.30 & 69.11 & 90 \\ 
            \hline
            7 & 117.0 & 116.9 & 38.73 & 15.12 & 13300 \\
            \hline
        \end{protocoltable}
        
    
    \subsection{Zpracované výsledky měření}

        \printfigure[Graf závislosti pulzů za sekundu na plošné hustotě pro $\beta$ zářič]{src/graf_5.png}{0.42}{sigma-beta}

        \begin{align*}
            V &= a \cdot b \cdot c = 0.1504 \cdot 0.1003 \cdot 0.00115 = 1.734 \cdot 10^{-5} \text{ m}^3 \neweq
            \rho &= \dfrac{m}{V} = \dfrac{0.03998}{1.734 \cdot 10^{-5}} = 2.306 \cdot 10^3 \text{ kg$\cdot$m$^{-3}$} \neweq
            \sigma &= \rho \cdot h = 2.306 \cdot 10^3 \cdot 0.00115 = 2.652  \text{ kg$\cdot$m$^{-2}$}
        \end{align*}

        \begin{equation}
            n = f(\sigma) = n_0 \cdot \text{e}^{{ -\mu \cdot h}} = n_0 \cdot \text{e}^{ -\mu_m  \cdot \sigma}
        \end{equation}

        \begin{equation*}
            n = 400.3 \cdot \text{e}^{-0.6836 \cdot \sigma} \rightarrow \boxed{\mu_m = 0.6836 \text{ m$^{-2} \cdot$s$^{-1}$}}
        \end{equation*}

        \begin{equation*}
            \mu = \mu_m \cdot \rho = 0.6836 \cdot 2.306 \cdot 10^{3} = 1.576 \cdot 10^3 m^{-1}
        \end{equation*}

        \begin{protocoltable}[Vypočtené objemy, plošné a objemové hustoty pro uvedené vzorky]{|C|C|C|C|C|C|C|C|}{ukol4-vypocty1}
            \hline
            Vzorek & 1 & 2 & 3 & 4  \\
            \hline
            $V$[m$^3$]&	$2.522 \cdot 10^{-6}$  & $3.738 \cdot 10^{-5}$ & $1.193 \cdot 10^{-5}$ & $1.734 \cdot 10^{-5}$ \\
            \hline
            $\rho$[kg$\cdot$m$^{-3}$] & $1.241 \cdot 10^{-3}$  & $1.321 \cdot 10^{3}$ & $1.677 \cdot 10^{3}$ & $2.306 \cdot 10^{3}$ \\
            \hline
            $\sigma$[kg$\cdot$m$^{-2}$] & 0.2482  & 3.304 & 1.677 & 2.652 \\
            \hline
            $n$[s$^{-1}$] & 326.3  & 30.67 & 119.0 & 57.37 \\
            \hline
            $\mu$[m$^{-1}$] & 848.4 & 903.3 & 1146 &	1576\\
            \hline
            \hline
            Vzorek & 5 & 6 & 7 & X \\
            \hline
            $V$[m$^3$]& $3.556 \cdot 10^{-5}$ & $1.126 \cdot 10^{-4}$ & $5.297 \cdot 10^{-4}$ & X\\
            \hline
            $\rho$[kg$\cdot$m$^{-3}$] & $9.662 \cdot 10^{2}$ & $6.135 \cdot 10^{2}$ & $2.854 \cdot 10^{1}$ & X \\
            \hline
            $\sigma$[kg$\cdot$m$^{-2}$] & 3.382 & 6.933 & 1.106 & X \\
            \hline
            $n$[s$^{-1}$] & 30.02 & 1.500 & 221.7 & X \\
            \hline
            $\mu$[m$^{-1}$] & 660.5 & 419.4 & 19.51 & X\\
            \hline
        \end{protocoltable}

        
        
    \subsection{Závěr}

\pagebreak

% Ukol 6 - 
\section{Úkol 6 - Nejistota určení hustoty}
    \subsection{Teoretický rozbor}
    \subsection{Postup měření}
    \subsection{Naměřené hodnoty}  
         \begin{protocoltable}[Naměřené rozměry vzorku 4]{|C|C|C|}{ukol6-rozmery}
            \hline
            a(délka)[mm] & b(šířka)[mm] & h(tloušťka)[mm]  \\
            \hline
            150.4 & 100.3 & 1.150 \\
            \hline
        \end{protocoltable}

         \begin{protocoltable}[Naměřené hmotnosti vzorku 5 pro vyhodnocení nejistoty typu A]{|C|C|C|C|C|C|}{ukol6-hmotnosti}
            \hline
            Měření[-] & 1 & 2 & 3 & 4 & 5 \\
            \hline
            $m$[g] &  39.18 & 39.19 & 39.16 & 39.16 & 39.17  \\
            \hline
            Měření[-]&  6 & 7 & 8 & 9 & 10 \\
            \hline
            $m$[g] & 39.17 & 39.15 & 39.16 & 39.20 & 39.17 \\
            \hline
        \end{protocoltable}
    
    \subsection{Zpracované výsledky měření}

        \begin{align*}
            u_B[a] = \dfrac{\Delta_a}{\chi} = \dfrac{5 \cdot 10^{-5}}{\sqrt{2}} = 7.212 \cdot 10^{-2}\text{ m}\\
            u_B[b] = \dfrac{\Delta_b}{\chi} = \dfrac{5 \cdot 10^{-5}}{\sqrt{2}} = 7.212 \cdot 10^{-2}\text{ m}\\
            u_B[h] = \dfrac{\Delta_h}{\chi} = \dfrac{5 \cdot 10^{-5}}{\sqrt{2}} = 7.212 \cdot 10^{-2}\text{ m}
        \end{align*}

        \begin{align*}
            u_C[V] = &\sqrt{ \left( \dfrac{\partial V}{\partial a} \cdot u_C(a) \right)^2 + \left( \dfrac{\partial V}{\partial b} \cdot u_C(b) \right)^2 + \left( \dfrac{\partial V}{\partial h} \cdot u_C(h) \right)^2} \neweq
            u_C[V] = &\sqrt{ \left( b \cdot h \cdot u_C(a) \right)^2 + \left( a \cdot h \cdot u_C(b) \right)^2 + \left( a \cdot b \cdot u_C(c) \right)^2 } \neweq
            u_C[V] = &\sqrt{ \left( 1.159 \cdot 10^{-4} \cdot 7.212 \cdot 10^{-2} \right)^2 + \left(  1.730 \cdot 10^{-4} \cdot 7.212 \cdot 10^{-2} \right)^2 + \dots } \neweq
            &\overline{\left( 1.510 \cdot 10^{-2} \cdot 7.212 \cdot 10^{-2} \right)^2}  \neweq
            u_C[V] = &\sqrt{ \left( 4.076 \cdot 10^{-7}\right)^2 + \left(  6.115 \cdot 10^{-7} \right)^2 + \left(  6.115 \cdot 10^{-9} \right)^2 } \neweq
            u_C[V] = &5.331\cdot 10^{-7} \text{ m$^3$}
        \end{align*}

        \begin{protocoltable}[Uncertainty budget pro určení objemu vzorku 5]{|C|C|C|C|}{ukol6-budget-objem}
            \hline
            Veličina & Hodnota & $\Delta_{MAX}$ & $\chi$ \\
            \hline
            $a$[m] & 0.1504 & $5 \cdot 10^{-5} $ m  & $\sqrt{2}$ \\
            \hline
            $b$[m] & 0.1003 &  $5 \cdot 10^{-5}$ m & $\sqrt{2}$\\
            \hline
            $h$[m] & 0.00115 &  $5 \cdot 10^{-5}$ m & $\sqrt{2}$\\
            \hline
            Veličina & $u_{A}$ & $\frac{\partial \rho}{\partial x_i}$ & $u_c$ \\
            \hline
            $a$[m] & 0 & $1.153 \cdot 10^{-4}$ m$^2$ & $4.076 \cdot 10^{-9}$ m$^3$\\
            \hline
            $b$[m] & 0 & $1.730 \cdot 10^{-4}$ m$^2$ & $6.115 \cdot 10^{-9}$ m$^3$\\
            \hline
            $h$[m] & 0 & $1.510 \cdot 10^{-2}$ m$^2$ & $5.331 \cdot 10^{-7}$ m$^3$\\
            \hline
            \multicolumn{2}{|c|}{$V = 1.734 \cdot 10^{-5}$ m$^3$} & \multicolumn{2}{c|}{$u_C = 5.331\cdot 10^{-7}$ m$^3$} \\
            \hline
        \end{protocoltable}

        \begin{equation*}
            \overline{m} = \dfrac{1}{n} \sum_{i = 0}^{n} m_i = \dfrac{1}{10} (0.03918 + \dots + 0.03917) = 0.03917\text{ kg}
        \end{equation*}

        \begin{align*}
            u_A[m] = &\sqrt{ \dfrac{1}{n(n-1)} \sum_{i = 0}^{n} (m_i - \overline{m})^2 } \neweq
            u_A[m] = &\sqrt{ \dfrac{1}{10(10-1)} ((0.03918 - 0.03917)^2 + \dots + (0.03917 - 0.03917)^2) } \neweq
            u_A[m] = &4.819 \cdot 10^{-6} \text{ kg}
        \end{align*}

        \begin{equation*}
            u_B[m] = \dfrac{\Delta_m}{\chi} = \dfrac{0.01}{\sqrt{3}} = 5.774 \cdot 10^{-6} \text{ kg}
        \end{equation*}

        \begin{equation*}
            u_C[m] = \sqrt{ u_A[m]^2 + u_B[m]^2  } = \sqrt{ (4.819 \cdot 10^{-6})^2 + (5.774 \cdot 10^{-6})^2 } = 7.520 \cdot 10^{-6} \text{ kg}
        \end{equation*}

        \begin{align*}
            u_C[\rho] = &\sqrt{ \left( \dfrac{\partial \rho}{\partial V} \cdot u_C(V) \right)^2 + \left( \dfrac{\partial \rho}{\partial m} \cdot u_C(m) \right)^2} \neweq
            u_C[\rho] = &\sqrt{ \left( \dfrac{-m}{V^2} \cdot u_C(V) \right)^2 + \left( \dfrac{1}{V} \cdot u_C(m) \right)^2  } \neweq
            u_C[\rho] = &\sqrt{ \left( -1.303 \cdot 10^{8} \cdot 5.331\cdot 10^{-7} \right)^2 + \left( 5.767 \cdot 10^{4} \cdot 7.520 \cdot 10^{-6} \right)^2  } \neweq
            u_C[\rho] = &\sqrt{ \left( -69.46  \right)^2 + \left( 0.4337 \right)^2  } = 69.46 \text{ kg$\cdot$m$^3$}
        \end{align*}

        \begin{equation*}
            U[\rho] = k_r \cdot u_C[\rho] = 2 \cdot 69.46 = 138.9 \text{ kg$\cdot$m$^3$}
        \end{equation*}

        \begin{equation*}
            \rho = \rho \pm U[\rho] = (2259 \pm 138.9) \text{ kg$\cdot$m$^3$} \rightarrow \boxed{ \rho = (2260 \pm 140) \text{ kg$\cdot$m$^3$}}
        \end{equation*}


        \begin{protocoltable}[Uncertainty budget pro určení objemové hustoty vzorku 5]{|C|C|C|C|}{ukol6-budget-hustota}
            \hline
            Veličina & Hodnota & $\Delta_{MAX}$ & $\chi$ \\
            \hline
            $V$[m$^3$] & $3.556 \cdot 10^{-5}$ & X & X \\
            \hline
            $m$[kg] & $3.917 \cdot 10^{-2}$  &  $1 \cdot 10^{-5}$ kg & $\sqrt{3}$\\
            \hline
            Veličina & $u_{A}$ & $\frac{\partial \rho}{\partial x_i}$ & $u_c$ \\
            \hline
            $V$[m$^3$] & 0 & $-1.303 \cdot 10^{8}$ kg & $-69.46$ kg$\cdot$m$^{-3}$ \\
            \hline
            $m$[kg] & $4.819 \cdot 10^{-6}$ kg & $5.767 \cdot 10^{4}$ m$^3$& $0.4337$ kg$\cdot$m$^{-3}$\\
            \hline
            \multicolumn{2}{|c|}{$\rho = 2259$ kg$\cdot$m$^{-3}$} & \multicolumn{2}{c|}{$u_C = 69.46 $ kg$\cdot$m$^{-3}$} \\
            \hline
        \end{protocoltable}
    \subsection{Závěr}

\pagebreak

% Ukol 7 - 
\section{Úkol 7 - Závislost počtu pulzů na hustotě Cs-137}
    \subsection{Teoretický rozbor}
    This is new
    \subsection{Postup měření}

    \subsection{Naměřené hodnoty}   
        \begin{protocoltable}[Naměřené počty pulzů za minutu při použití gamma zářiče pro určené vzorky]{|C|C|C|C|C|C|C|C|}{ukol7-mereni}
            \hline
            Vzorek & 1 & 2 & 3 & 4 & 5 & 6 & 7 \\
            \hline
            $n$[min$^{-1}$] & 177 & 167 & 202 & 171 & 170 & 155 & 189 \\
            \hline
        \end{protocoltable}
    \subsection{Zpracované výsledky měření}

         \printfigure[Graf závislosti pulzů za sekundu na plošné hustotě pro $\gamma$ zářič]{src/graf_7.png}{0.45}{sigma-gamma}
    \subsection{Závěr}

\pagebreak

% Velký závěr
\section{Závěr}
    \cite{navod}

\pagebreak


% Seznam přístrojů
\section{Seznam použitých přístrojů}

\pagebreak

% Reference
\bibliography{ref}

\end{document} % Konec dokumentu
