\documentclass{protokol}
\usepackage{array}
\usepackage{tabularx}
\usepackage{environ}
%------------------- Zde vyplňte údaje -------------------------------
\autor{Václav Horáček}
\autorID{256296}
\autorr{Jan Holík}
\autorrID{256295}
\rocnik{3}
\merenodne{14.\,10.\,2025}
\nazev{Měření ionizujícího záření}
\predmet{Snímače}
%=====================================================================
\begin{document}
\maketitle                  % Vygeneruje titulní stránku podle vyplněných údajů
%\tableofcontents\newpage   % Vygeneruje obsah
%------------------- Zde začíná samotný dokument ---------------------

% command pro psani promennych k rovnicim
\newenvironment{conditions}
  {\par\vspace{\abovedisplayskip}\noindent\begin{tabular}{>{$}l<{$} @{${}-{}$} l}}
  {\end{tabular}\par\vspace{\belowdisplayskip}}

% Zadává
\bibliographystyle{acm}

% Uprava tabulek
\def\arraystretch{1.3}
\setlength{\headheight}{15pt}
\renewcommand{\sectionmark}[1]{\markboth{#1}{}}

\newcolumntype{C}{>{\centering\arraybackslash}X}

% FUNKCE PRO GENEROVANI TABULEK 
%   1. argument popisek
%   2. argument format
%   3. argument label
%   Do tela psat data a \hline
\NewEnviron{protocoltable}[3][Tabulka]{%
    \begin{table}[!h]
        \centering
        \caption{#1}\label{tab:#3}
        \vspace{0.3cm}
        \begin{tabularx}{\textwidth}{#2}
            \BODY
        \end{tabularx}
    \end{table}
}

% FUNKCE PRO TISK OBRAZKU
%  1. argument titulek
%  2. argument cesta napr. src\neco.png
%  3. argument scale - velikost rozsah 0.0-1.0
%  4. argument label
\newcommand{\printfigure}[4][Obrazek]{%
    \begin{figure}[!h]
        \centering
        \includegraphics[scale=#3]{#2}
        \caption{#1}
        \label{obr:#4}
    \end{figure}
}


\section{ZADÁNÍ}\label{kap:zadani}
    \begin{enumerate}
        \item Body zadání 1.-6. realizujte s $\beta$-zářiči Sr-90. Proveďte základní dozimetrická měření:
        
            \begin{enumerate}
                \item Změřte hodnotu přirozeného pozadí ionizujícího záření pomoci dozimetru Gamma-Scout.
                
                \item Zvýšené hodnoty záření způsobené $\beta$-zářičem během laboratorního cvičení pomocí dozimetru Gamma-Scout.
            
                \item V tabulce přehledně porovnejte hodnoty naměřené dozimetrem s hodnotami odpovídajícími hygienickým limitům.
            \end{enumerate}
           
        \item Proměřte impulzovou charakteristiku GM trubice, určete délku „plateau“ \linebreak a jeho strmost.

        \item Určete mrtvou dobu GM trubice, napájecí napětí volte ze středu plateau, porovnejte výsledky získané metodou dvou zářičů a přímým měřením na osciloskopu.

        \item Určete vliv stínící přepážky při měření závislosti $I = f(d)$, kde $d$\linebreak je tloušťka materiálu. Závislost vyneste do grafu. Určete součinitel zeslabení $\mu$  \linebreak a polotloušťku $d_{1/2}$.

        \item Určete hmotnostní koeficient útlumu $\mu_m$ charakterizující útlum ionizujícího záření v předložených materiálech, porovnejte hodnotu získanou výpočtem pro každý materiál s hodnotou stanovenou z grafu závislosti funkce intenzity \linebreak plošné hustotě.

        \item Stanovte nejistotu určení hustoty pro jeden materiál.

        \item S $\gamma$-zářičem Cs-137 změřte četnost impulzů pro sadu vzorků s různou hustotou \linebreak a porovnejte závislost počtu pulzů na hustotě s měřením s $\beta$-zářičem \linebreak z bodu 5.

    \end{enumerate}

\pagebreak 

% Ukol 1 - 
\section{Úkol 1 - Základní dozimetrická měření}
    \subsection{Teoretický rozbor}
    \subsection{Postup měření}
    \subsection{Naměřené hodnoty}   
    \subsection{Zpracované výsledky měření}
    \subsection{Závěr}

\pagebreak

% Ukol 2 - 
\section{Úkol 2 - Impulzová charakteristika GM trubice}
    \subsection{Teoretický rozbor}
    \subsection{Postup měření}
    \subsection{Naměřené hodnoty}   
    \subsection{Zpracované výsledky měření}
    \subsection{Závěr}

\pagebreak

% Ukol 3 - 
\section{Úkol 3 - Mrtvá doba GM trubice}
    \subsection{Teoretický rozbor}
    \subsection{Postup měření}
    \subsection{Naměřené hodnoty}   
    \subsection{Zpracované výsledky měření}
    \subsection{Závěr}

\pagebreak

% Ukol 4 - 
\section{Úkol 4 - Vliv stínící přepážky}
    \subsection{Teoretický rozbor}
    \subsection{Postup měření}
    \subsection{Naměřené hodnoty}   
    \subsection{Zpracované výsledky měření}
    \subsection{Závěr}

\pagebreak

% Ukol 5 - 
\section{Úkol 5 - Hmotnostní koeficient útlumu}
    \subsection{Teoretický rozbor}
    \subsection{Postup měření}
    \subsection{Naměřené hodnoty}   
    \subsection{Zpracované výsledky měření}
    \subsection{Závěr}

\pagebreak

% Ukol 6 - 
\section{Úkol 6 - Nejistota určení hustoty}
    \subsection{Teoretický rozbor}
    \subsection{Postup měření}
    \subsection{Naměřené hodnoty}   
    \subsection{Zpracované výsledky měření}
    \subsection{Závěr}

\pagebreak

% Ukol 7 - 
\section{Úkol 7 - Závislost počtu pulzů na hustotě Cs-137}
    \subsection{Teoretický rozbor}
    \subsection{Postup měření}
    \subsection{Naměřené hodnoty}   
    \subsection{Zpracované výsledky měření}
    \subsection{Závěr}

\pagebreak

% Velký závěr
\section{Závěr}
    \cite{navod}

\pagebreak


% Seznam přístrojů
\section{Seznam použitých přístrojů}

\pagebreak

% Reference
\bibliography{ref}







%------------------------------ Ukázky -------------------------------
\begin{comment} % Komentář bloku
    \newpage % Začátek na nové stránce
        
    % Ukázka obrázku
    \begin{figure}[!h]
        \centering
        \includegraphics[scale=0.3]{UAMT_color_CMYK_CZ.pdf}
        \caption{Logo UAMT}
        \label{obr:logo}
    \end{figure}

    % Ukázka tabulky
    \begin{table}[!h]
        \caption{Ukázka tabulky}
        \begin{center}
            \begin{tabular}{ |c|c|c| } 
                \hline
                Jméno & Příjmení & ID \\ 
                \hline\hline
                Jan   & Novák    & 16 \\ 
                \hline
                Petr  & Novák    & 23 \\ 
                \hline
            \end{tabular}
        \end{center}
        \label{tab:ukazka}
    \end{table}
    
    % Ukázka rovnice
    \begin{equation} \label{rov:pythagor}
        c^{2} = a^{2} + b^{2}
    \end{equation}
    
    % Ukázka číslovaného výčtu
    \begin{enumerate}
        \item Úkol č.1
        \item Úkol č.2
    \end{enumerate}
    
    % Ukázka výčtu
    \begin{itemize}
        \item Přístroj č.1
        \item Přístroj č.2
    \end{itemize}
    
    % Příklad využití odkazů v textu:
    Na obrázku \ref{obr:logo} se nachází \dots\\
    V tabulce \ref{tab:ukazka} je uveden \dots\\
    Rovnice \eqref{rov:pythagor} definuje vztah pro Pythagorovu větu.
    V kapitole \nameref{kap:zadani} \dots
    
\end{comment}

\end{document} % Konec dokumentu
